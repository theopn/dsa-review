%! TEX root = ../dsa-mini-textbook.tex

\textit{Algorithms} are any well-defined computational procedures that take some value(s) as input and produce more value(s) as output. They are \textbf{effective}, \textbf{precise}, and \textbf{finite}. There are several ways to analyze the runtime of an algorithm.

\section{Power Law}

\begin{enumerate}
  \item For the algorithm, get a table for the input size $n$ and the runtime $T(n)$.
    \begin{center}
      \begin{tabular}{ | c | c | } % c for center, l for left-align
        \hline
        $n$ & $T(n)$ \\
        \hline
        250 & 0.0 \\
        500 & 0.012 \\
        1000 & 0.0954 \\
        2000 & 0.7727 \\
        4000 & 6.1664 \\
        \hline
      \end{tabular}
    \end{center}
  \item Make sure that the data plots are valid for the power law analysis by checking the following properties:
    \begin{itemize}
      \item \textbf{have enough data plots.} For instance, if there are only two data plots, you should not make the power law conjecture.
      \item \textbf{fits the power law.} You can verify this by finding the ratio between data plots.

        \begin{center}
          \begin{tabular}{ | c | c | c | }
            \hline
            $n$ & $T(n)$ & ratio \\
            \hline
            250 & 0.0 & -- \\
            500 & 0.012 & -- \\
            1000 & 0.0954 & 0.0954 / 0.012 = 7.95 \\
            2000 & 0.7727 & 0.7727 / 0.0954 = 8.10 \\
            4000 & 6.1664 & 6.1664 / 0.7727 = 7.98 \\
            \hline
          \end{tabular}
        \end{center}
    \end{itemize}

    For the ratios we found, //TODO
\end{enumerate}

\section{Runtime Expressions}

\section{Asymptotic Runtime Analysis}

\section{Recursive Relationship}
