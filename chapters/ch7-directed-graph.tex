%! TEX root = ../dsa-review.tex

\noindent The overall constructor is identical to that of undirected graph except that for an edge directing $u \rightarrow v$, only $v$ is added to the adjacency list/matrix of $u$ and not the other way around.

\noindent I also added a recursive implementation of functional \textsc{DFS} method that modifies external \textsc{visited} and \textsc{path} arrays.

\begin{python}
class Graph:
    # e.g.,: Graph(n = 3, edges = [[0,1],[1,2],[0,2]])
    # creates a graph with 3 nodes indexed 0, 1, 2,
    # with 0 -> 1, 0 -> 2, 1 -> 2
    def __init__(self, n: int, edges: list[list[int]]):
        self.n = n
        self.edges = edges

        self.adjList = [[] for _ in range(n)]
        for u, v in edges:
            self.adjList[u].append(v)

        self.adjMatrix = [[float("inf") for _ in range(n)] for _ in range(n)]
        for i in range(v):
            self.adjMatrix[i][i] = 0
        for u, v in edges:
            self.adjMatrix[u][v] = 1

    def dfs(self, v, visited, path):
        visited[v] = True
        path.append(v)

        for w in self.adjList[v]:
            if not visited[w]:
                self.dfs(w, visited, path)
\end{python}

\section{Strong Connectivity}

A directed graph is strongly connected if every vertex is reachable from any other vertex.
To test the strong connectivity, use the following algorithm, which takes $O(|V| + |E|)$.

\noindent \hrulefill
\begin{algorithmic}[1]
  \Function{is-strongly-connected}{$G$}
    \State Pick a vertex $v$ in $G$
    \State Perform \textsc{DFS($G, v$)}
    \State If there is an unvisited vertex, return false
    \State Compute $G^T$ (transpose of $G$, flip all the edges)
    \State Perform \textsc{DFS($G^T, v$)}
    \State IF there is an unvisitied vertex, return false
    \State return true
  \EndFunction
\end{algorithmic}
\noindent \hrulefill

\subsection{Strongly Connected Components and Kosaraju's Algorithm}

$O(|V| + |E|)$

\begin{enumerate}
  \item Perform DFS with a global stack. For each run, append the them in the reversal order of how they would be appended to the path in the regular DFS (this is essentially picking a vertex with the lowest in-degrees)
  \item Iterating through the stack, run the DFS on a transposed graph
  \item The set of visited vertex from this run will form a strongly connected component
\end{enumerate}

\begin{python}
    def kojaraju(self):
        # Run DFS to make a stack of vertices based on exit time
        visited = [False] * self.n
        indegreePriorityStack = []

        def visit(v):
            visited[v] = True

            for w in self.adjList[v]:
                if not visited[w]:
                    visit(w)
            indegreePriorityStack.append(v)

        for v in range(self.n):
            if not visited[v]:
                visit(v)

        # DFS on the transposed graph, in the order of exit time
        transposedG = Graph(self.n, [[v, u] for u, v in self.edges])
        visitedTranspose = [False] * self.n
        scc = []
        while indegreePriorityStack:
            v = indegreePriorityStack.pop()
            if not visitedTranspose[v]:
                path = []
                transposedG.dfs(v, visitedTranspose, path)
                scc.append(sorted(path))

        return scc
\end{python}

\section{Articulation Points}

An articulation point is a vertex such that removing it from the graph increases the number of connected components.
This is applicable for undirected graph as well.

\noindent \hrulefill
\begin{algorithmic}[1]
  \Function{AP}{$G, s, \textrm{discovery}$} \Comment{$s$ is the starting vertex}
  \State Mark $s$ as visited
  \State $\mathrm{discovery}[s] \gets d$
  \State $\mathrm{low}[s] \gets d$
  \For{$w$ adj to $s$}
    \If{$v$ is not visited}
      \Call{AP}{$g$, $s$, $d$}
    \EndIf
    \State $\mathrm{low}[s] \gets \Call{min}{\mathrm{low}[s], \mathrm{low}[w]}$
    \If{$\mathrm{low}[s] \geq \mathrm{discovery}[s]$}
      \If{$s$ is not root or $s$ has more than 1 child in the DFS tree}
        \State \Call{Print}{$s$ is the AP}
      \EndIf
    \EndIf
  \EndFor
  \EndFunction
\end{algorithmic}
\noindent \hrulefill

\section{Directed Acyclic Graphs}

DAG is a directed graph without a cycle.

\subsection{Topological Sort}

\noindent \hrulefill
\begin{algorithmic}[1]
  \Function{kahn-topological-sort}{$G$} \Comment{$G$ is the graph object}
    \State $H \gets$ a specialized graph able to keep track of in-degrees of vertices, initialized with $G$
    \State $Ordering \gets$ array of size $|V|$
    \State $n \gets 0$
    \item[]
    \While{$H$ is not empty}
      \State $v \gets$ a vertex with in-degree 0 from $H$
      \State $Ordering[n] = v$
      \State $P.\text{remove}(v \text{ and all connect edges})$
      \State $n \gets n + 1$
    \EndWhile
  \EndFunction
\end{algorithmic}
\noindent \hrulefill

\noindent If $G$ is not a DAG (i.e., contains cycle(s)), $H$ will still have edge(s) left in it because vertices in a cycle will never reach in-degree 0.
This means that you can check if a graph is DAG by running the topological sort and asserting $Ordering.\text{length} = |V|$

\noindent Here is an actual implementation of the algorithm using queue.

\begin{python}
    def kahnTopologicalSort(self):
        indeg = [0] * self.n

        for u, v in self.edges:
            indeg[v] += 1

        q = deque()
        for v in range(self.n):
            if indeg[v] == 0:
                q.append(v)

        ordering = []
        while q:
            u = q.pop()
            ordering.append(u)

            for w in self.adjList[u]:
                indeg[w] -= 1  # "removing" the edge
                if indeg[w] == 0:
                    q.append(w)

        # if we do not have all the nodes in the ordering list,
        # it means that the graph is not DAG.
        # It is because if the graph is DAG, all nodes will eventually have
        # zero indegree after edge removal
        return [] if len(ordering) != self.n else ordering
\end{python}
