%%%%%%%%%%%%%%%%%%%%%%%%%%%%%%%%%%%%%%%%%%%%%%%%%%%%%%%%%%%%%%%%%%%%%%%%%%%%%%%%
%%%%%%%%%% Author: Theo Park
%%%%%%%%%%%%%%%%%%%%%%%%%%%%%%%%%%%%%%%%%%%%%%%%%%%%%%%%%%%%%%%%%%%%%%%%%%%%%%%%
\documentclass{report}
\usepackage[utf8]{inputenc}

\usepackage{gruvbox-colorscheme}

\usepackage{algpseudocodex} % best algorithm package
\usepackage{amsmath} % \text
\usepackage{amssymb} % \therefore
\usepackage{forest} % Treeeeeeeee
\usepackage{hyperref} % hyperlink for toc
\usepackage{import} % better \input \include
\usepackage{multicol} % figures side by side
\usepackage[default]{sourcesanspro} % font
\usepackage{soul} % \st
\usepackage{tikz} % you know what tikz is
\usetikzlibrary{calc,shapes.multipart,chains,arrows}
\usepackage{minted} % Python code block

% Pkg used for both header and footer
\usepackage{fancyhdr}
% Header
\topmargin=-0.45in
\evensidemargin=0in
\oddsidemargin=0in
\textwidth=6.5in
\textheight=9.0in
\headsep=0.25in

% Footer
\renewcommand{\footrulewidth}{0.5pt} % Footer line thickness
\rfoot{\small{\textit{By Theo Park, based on Purdue Fall 2022 CS251}}}

% Title page
\title{Theo's DSA Review}
\author{Theo Park}
\date{Compiled: \today}

\begin{document}

\maketitle

\pagestyle{fancy}

% TOC

\tableofcontents

\chapter*{Preface}

\noindent This was a small project of mine while I was taking gap years, and I had a great pleasure of writing this.
I hoped to refine this further with more examples and explanations, but it is time to go back to school and learn new things.

\noindent The materials in this review are loosely based on

\begin{itemize}
  \item Purdue University CS251: Data Structures \& Algorithms, taught in the fall of 2022 by Prof. Andres Bejarano -- Thanks, Prof. Bejarano
  \item Introduction to Algorithms Third Edition by "CLRS"
\end{itemize}

\noindent Most of the algorithms in the document are written in Python to provide real-life implementations.
You can find the sources in the \textsc{code} directory of the GitHub repository.
Some test cases are generated with generative AI models, either Windsurf model using Neocodeium Neovim plugin or OpenAI ChatGPT.
However, AI uses were restricted to the generation of test cases and all codes are direct Python translation of pseudocodes in CLRS.

\noindent For more details regarding compiling, see \textsc{README.md} in the GitHub repository.

\noindent Constructive criticism are always welcome, but I do not take any responsibility for any errors in the document.

\noindent \textbf{Every files in the repository, including but not limited to the \LaTeX source files are licensed under the following.}
\begin{verbatim}
MIT License

Copyright (c) 2022-present Theo Park

All rights reserved.

Permission is hereby granted, free of charge, to any person obtaining a copy
of this software and associated documentation files (the "Software"), to deal
in the Software without restriction, including without limitation the rights
to use, copy, modify, merge, publish, distribute, sublicense, and/or sell
copies of the Software, and to permit persons to whom the Software is
furnished to do so, subject to the following conditions:

The above copyright notice and this permission notice shall be included in all
copies or substantial portions of the Software.

THE SOFTWARE IS PROVIDED "AS IS", WITHOUT WARRANTY OF ANY KIND, EXPRESS OR
IMPLIED, INCLUDING BUT NOT LIMITED TO THE WARRANTIES OF MERCHANTABILITY,
FITNESS FOR A PARTICULAR PURPOSE AND NONINFRINGEMENT. IN NO EVENT SHALL THE
AUTHORS OR COPYRIGHT HOLDERS BE LIABLE FOR ANY CLAIM, DAMAGES OR OTHER
LIABILITY, WHETHER IN AN ACTION OF CONTRACT, TORT OR OTHERWISE, ARISING FROM,
OUT OF OR IN CONNECTION WITH THE SOFTWARE OR THE USE OR OTHER DEALINGS IN THE
SOFTWARE.
\end{verbatim}

% Chapter 1

\chapter{Runtime Analysis}
\import{chapters}{ch1-runtime-anal.tex}

% Chapter 2

\chapter{Intro to Data Structures}
\import{chapters}{ch2-ds.tex}

% Chapter 3

\chapter{Sorting Algorithms}
\import{chapters}{ch3-sorting-algo.tex}

% Chapter 4

\chapter{Hash Tables}

\section{Division Method}

\section{Multiplication Method}

\section{Collision}

\subsection{Chaining}

\subsection{Open Addressing}

% Chapter 5

\chapter{Search Tree}

\section{Binary Search Tree and Its Limit}

\section{2-3 Tree}

\section{Red-Black Tree}

\section{Left-Leaning Red-Black Tree}

\subsection{Deletion in LLRBT}

% Chapter 6

\chapter{Undirected Graph}

\import{chapters}{ch6-undirected-graph.tex}

% Chapter 7

\chapter{Directed Graphs}

\import{chapters}{ch7-directed-graph.tex}

% Chapter 8

\chapter{Weighted Graphs}

\import{chapters}{ch8-weighted-graph.tex}

% Chapter 9

\chapter{Strings}

\import{chapters}{ch9-str.tex}

\end{document}

